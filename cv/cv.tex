%%% A template to produce a nice-looking Curriculum Vitae.
%%% Original by Kieran Healy <kjhealy@gmail.com>
%%%
%%% ------------------------------------------------------------------------
%%% Requirements (should be included in a modern tex distribution):
%%% ------------------------------------------------------------------------
%%% xelatex
%%% fontspec.sty
%%% hyperrref.sty
%%% xunicode.sty
%%% color.sty
%%% url.sty
%%% fancyhdr.sty
%%%
%%% ------------------------------------------------------------------------
%%% Optional
%%% ------------------------------------------------------------------------
%%% git
%%% vc.sty
%%% revnum.sty
%%% Fonts
%%%
%%% ------------------------------------------------------------------------
%%% Note
%%%------------------------------------------------------------------------
%%% Because this is a hand-tweaked file, be on the look out for \medksip,
%%% \bigskip and \newpage commands here and there, which are used to balance
%%% the layout or avoid widows & orphans, etc. You should of course add or
%%% remove these as needed.
%%%------------------------------------------------------------------------

\documentclass[9pt]{article}

%%%------------------------------------------------------------------------
%%% Metadata
%%%------------------------------------------------------------------------

%% Change as needed. Or just add me as a coauthor. Only some of these are
%% used below in the hyperref declaration and address banner section.
\def\myauthor{Aleksandar Prokopec}
\def\mytitle{Vita}
\def\mycopyright{\myauthor}
\def\mykeywords{}
\def\mybibliostyle{plain}
\def\mybibliocommand{}
\def\mysubtitle{}
\def\myaffiliation{Google Inc.}
\def\myaddress{Feldblumenstrasse 100, 8134 Adliswil}
\def\myemail{aleksandar.prokopec@gmail.com}
\def\myweb{http://axel22.github.io}
%\def\myfax{}
\def\myphone{+41 78 9489 347}
\def\myversion{}
\def\myrevision{}


\def\myaffiliation{Google Inc.}
\def\myauthor{Aleksandar Prokopec}
\date{} % not used (revision control instead)
\def\mykeywords{Aleksandar, Prokopec, Aleksandar Prokopec, Vita, CV, Resume, Scala,
Programming Languages, Parallelism, Concurrency, Distributed Programming}

%%%------------------------------------------------------------------------
%%% Git version tracking
%%%------------------------------------------------------------------------

%% If you don't use git or the vc package (from CTAN), comment this out.
%% If you comment it out, be sure to remove the \rfoot comment below, too.
%\immediate\write18{sh ./vc}
%\input{vc}

%%%------------------------------------------------------------------------
%%% Required style files
%%%------------------------------------------------------------------------

\usepackage{url,fancyhdr}
\usepackage[ampersand]{easylist}
%%\usepackage{revnum}
% for reverse-numbered publications (revnumerate environment) if needed.

%% needed for xelatex to work
\usepackage{fontspec}
\usepackage{xunicode}

%% color for the links
% \usepackage[usenames,dvipsnames]{color}
\usepackage[usenames,dvipsnames]{xcolor}
\definecolor{LinkColor}{HTML}{264FA0}

%% hyperlinks
\usepackage[xetex,
  colorlinks=true,
  urlcolor=LinkColor,
  plainpages=false,
    pdfpagelabels,
    bookmarksnumbered,
    pdftitle={\mytitle},
    pagebackref,
    pdfauthor={\myauthor},
    pdfkeywords={\mykeywords}
    ]{hyperref}

\usepackage{marvosym}

% \usepackage{showframe}
\usepackage[width=4.825in,top=1.7in]{geometry}

\usepackage{fontawesome}

%%%------------------------------------------------------------------------
%%% Document
%%%------------------------------------------------------------------------
\begin{document}

%% Choose fonts for use with xelatex
%% Minion and Myriad are widely available, from Adobe.
%% Pragmata is available to buy at http://www.fsd.it/fonts/pragma.htm
%% and is worth every penny. Any good monospace font will work fine, though.
%% Consolas or inconsolata are good alternatives.
\setromanfont[Mapping={tex-text},Numbers={OldStyle},Ligatures={Common}]{Minion Pro}
\setsansfont[Mapping=tex-text,Colour=AA0000]{Myriad Pro}
\setmonofont[Mapping=tex-text,Scale=0.9]{Consolas}


%%%------------------------------------------------------------------------
%%% Local commands
%%%------------------------------------------------------------------------

%% Marginal header
%% Note: as the document goes on you may need to introduce a (gradually increasing)
%% \vspace element to keep the marginal header pleasingly aligned with the first
%% item in the body text. Like this: \marginhead{{\vskip 0.4em}Grants}, or
%% \marginhead{{\vskip 0.8em}Service}. Experiment as needed.
\newcommand{\marginhead}[1]
{\marginpar{\textsf{{\normalsize\vspace{-1em}\flushright #1}}}}

\newcommand{\dates}[1]{\hfill \emph{#1}}

%% custom ampersand (font consistent with the one chosen above)
\newcommand{\amper}
{{\fontspec[Scale=.95,Colour=AA0000]{Minion Pro Medium}\selectfont\&\,}}

%% No bullets on labels
\renewcommand{\labelitemi}{~}

%% Custom hanging indent for vita items
\def\ind{\hangindent=1 true cm\hangafter=1 \noindent}
%\def\ind{\hangindent=18pt\hangafter=1 \noindent}
\def\labelitemi{~}
\renewcommand{\labelitemii}{~}

%%%------------------------------------------------------------------------
%%% Page layout
%%%------------------------------------------------------------------------
\pagestyle{fancy}
\renewcommand{\headrulewidth}{0pt}
\fancyhead{}
\fancyfoot{}
\rfoot{{\scriptsize\thepage}}

% \setlength{\headsep}{12pt}
\textheight=580pt
\raggedbottom
\thispagestyle{fancy}

%% git revision control footer
%\rfoot{\texttt{\scriptsize \VCRevision\ on \VCDateTEX}}
% git revision info inserted via external script -- see docs for vc package for details.
%comment out this line if you're not using vc, also remove the \input{vc} line above.


%%%------------------------------------------------------------------------
%%% Address and contact block
%%%------------------------------------------------------------------------
% \begin{absolutelynopagebreak}
\begin{minipage}[t]{1.55in}
 \flushright \href{http://www.google.com/about/company/}{Google Zürich}
   \\ \vspace{-0.03in} Brandschenkestrasse 110
   \\ \vspace{-0.03in} 8002 Zürich
   \\ \vspace{-0.03in} Switzerland

\end{minipage}
\hfill
%\begin{minipage}[t]{0.0in}
% dummy (needed here)
%\end{minipage}
\hfill
\begin{minipage}[t]{2.7in}
  \flushright Phone: \myphone
  \\ \vspace{-0.03in} Address: \myaddress
  \\ \vspace{-0.03in} E-mail: \texttt{\href{mailto:\myemail}{\myemail}}
  \\ \vspace{-0.03in} Website: \texttt{\href{\myweb}{\myweb}}
\end{minipage}


\medskip
\medskip
\medskip

%% Name
\noindent{\huge {\textsc{aleksandar prokopec}}}
\reversemarginpar

\medskip

%% Citizenship
\medskip
\marginhead{Personal \newline Information}

\noindent Birth date: 6. August 1985.
\newline\noindent Slovenian, married
\newline\noindent Swiss Permit B (2014 -- 2019)

\bigskip


%% Summary
\marginhead{Summary}

\noindent
I am a computer science researcher with 7 years of academic
and industrial experience.
I published 12 peer-reviewed research publications,
and led multiple software projects,
supervised EPFL-hosted open source projects with external funding from Google,
and organized a highly successful Coursera massive open online course
on reactive programming.
I have participated in several international collaborations
with top universities and industrial partners.
I participated on 17 industrial and research conferences and meetups,
held 6 invited talks,
and authored a textbook on concurrent programming in Scala.
Since 2014, I am employed as a software engineer at Google.
\newline


%% Education
\marginhead{Education}

\noindent{\bf \em École Polytechnique Fédérale de Lausanne},
\emph{Lausanne, Switzerland} \vspace{0.01in} \dates{2009 -- 2014}
\newline Ph.D. in Computer Science
\newline Advisor: Martin Odersky
\newline Committee: Douglas Lea, Erik Meijer, Viktor Kuncak, Ola Svensson
\bigskip

\noindent{\bf \em Faculty of Electrical Engineering and Computing},
\emph{Zagreb, Croatia} \vspace{0.01in}
\dates{2004 -- 2009}
\newline M.A. in Computer Science
\newline Advisor: Marin Golub
\bigskip

%% Professional Experience
\medskip
\marginhead{Professional \newline Experience}

\noindent {\bf Software Engineer}, {\bf \em Google Inc.}, \emph{Zürich, Switzerland}
\vspace{0.01in} \dates{2014 -- 2016}
\newline\noindent Product area: Geo
\begin{easylist}[itemize]
& Maintaining a distributed logging infrastructure for Google Maps products.
& Working on a massively distributed system used to analyze Geo product usage.
& Led the implementation of a real-time distributed pipeline for spam detection.
  \newline Awarded a \emph{Google Spot Bonus} for this effort.
& Designed and implemented a test automation suite for product documentation.
& Main organizer of the team-internal monthly Hackathon events.
  \newline Awarded a \emph{Google Peer Bonus} for this effort.
\end{easylist}
\bigskip

\noindent {\bf Scala Open Source Developer},
{\bf \em Scala Team}, \emph{Lausanne, Switzerland}
\vspace{0.01in} \dates{2009 -- 2016}
\vspace{0.05in}
\begin{easylist}[itemize]
& Designed and implemented \emph{Scala Coroutines} -- language extension
  for first-class coroutines, used to facilitate asynchronous programming.
& Designed and implemented \emph{Parallel Collections} -- support for data-parallel
  programming (part of the Scala programming language since 2011).
& Core part of the \emph{Futures and Promises} working group,
  aimed to design the support for asynchronous programming
  (part of Scala programming language since 2012).
& Maintained and improved the Scala compiler and the Scala standard library.
& Engaged in various open-source activities: ScalaDays conference organization,
  Scala Workshop program committee member, Scala Improvement Proposal process,
  online education (MOOCs) and documentation, etc.
\end{easylist}
\bigskip

%% Collaborations
\textheight=580pt
\marginhead{{\vskip 0.3em}Collaborations}
% \medskip

\vspace{0.05in}

\noindent {\bf FORTH Institute of Computer Science (Greece)},
\dates{2015 -- now}
\newline\noindent
{\em Computer Architecture and VLSI Systems Laboratory}
\dates{}
\newline\noindent Research collaboration aimed at developing a novel high-performance
\newline\noindent concurrent data structures for traditional embedded systems.
\medskip

\noindent {\bf Typesafe (USA)},
\dates{2011 -- now}
\newline\noindent Collaborating on maintenance, development, technology and innovation
\newline\noindent exchange related to the  Scala project
                  (\textasciitilde30 Scala developers).
\medskip

\noindent {\bf Akka team (Sweden)},
\dates{2011 -- 2012}
\newline\noindent Member of the working group (\textasciitilde10 people) that developed
                  a unifying
\newline\noindent asynchronous programming framework for Scala.
\medskip

\noindent {\bf Stanford University (USA)},
\dates{2011 -- 2013}
\newline\noindent
{\em Pervasive Parallelism Laboratory}
\dates{}
\newline\noindent Collaborated on the LMS and Delite compiler frameworks and runtimes
\newline\noindent for parallel embedded domain-specific languages
                  (\textasciitilde10 people). Designed
\newline\noindent high-performance data structures and collections frontends.
\medskip


%% Teaching Experience
\medskip
\marginhead{Teaching \newline Experience}

% \begin{minipage}{\linewidth}
\noindent {\bf External Lecturer, Co-Organizer},
{\em Reactive Programming and Parallelism}
\dates{2015}
\newline\noindent Co-organized, prepared materials and exercises, led teaching staff
                  (\textasciitilde7 people)
\newline\noindent on the undergraduate course on parallel, distributed, and asynchronous
\newline\noindent programming at EPFL, \textasciitilde90 students.
\bigskip

\noindent {\bf Lecturer, Co-Organizer},
{\em Parallel Programming and Data Analysis}
\dates{2015}
\newline\noindent Coursera MOOC on parallel and asynchronous programming.
\bigskip

\noindent {\bf Lead Organizer},
{\em Principles of Reactive Programming}
\dates{2013}
\newline\noindent Coursera MOOC on reactive programming in Scala, with 2 iterations
\newline\noindent and >60,000 participants so far.
\vspace{0.05in}
\begin{easylist}[itemize]
& Coordinated a team of three lecturers during recording and
\newline lecture material production.

& Led graduate student teaching staff (\textasciitilde8 people),
\newline directed content production, designed and implemented exercise
\newline materials, managed the production of lecture videos,
\newline organized community TAs on Coursera.

& Received the \emph{EPFL IC Teaching Award} for this effort.
\end{easylist}
\bigskip

\noindent {\bf Teaching Assistant}, {\em Functional Programming in Scala}
\dates{2010-2013}
\newline\noindent Required EPFL undergraduate course on functional programming
\newline\noindent (\textasciitilde160 students).
\medskip
\bigskip
\bigskip
\bigskip
\bigskip
\bigskip
\bigskip
\bigskip
\bigskip
\bigskip
\bigskip


% \noindent {\bf Teaching Assistant}, {\em Programming Principles} \dates{2011, 2014}
% \newline\noindent Required EPFL Undergraduate course on functional and logic programming
% \newline\noindent (\textasciitilde160 students)
% \bigskip

\medskip

%% Research Interests
\textheight=580pt
\marginhead{{\vskip 0.3em}Research \newline Interests}
% \medskip

\noindent
My focus is design and implementation of frameworks,
programming languages, and runtime support for concurrent, parallel and
distributed software development.
I proposed and implemented novel persistent, concurrent and incremental
data structures that support these programming paradigms.
I use the Scala programming language as both the underlying development platform
and research vehicle.
\bigskip

% Programming language support for concurrent and distributed programming; \\type systems; non-standard uses of types for data-centric programming and big data; language and library design


% \bigskip

\marginhead{{\vskip 0.4em}Thesis}
\medskip

\noindent\href{http://infoscience.epfl.ch/record/200977/files/EPFL_TH6264.pdf}
{\bf Data Structures and Algorithms for Data-Parallel Computing }
\dates{EPFL 2014}
\newline
\noindent\href{http://infoscience.epfl.ch/record/200977/files/EPFL_TH6264.pdf}
{\bf in a Managed Runtime}
\newline
\noindent Aleksandar Prokopec
\bigskip


\marginhead{{\vskip 0.4em}Books}
\medskip

\noindent\href{http://www.amazon.com/Learning-Concurrent-Programming-Aleksandar-Prokopec/dp/1783281413/}
{\bf Learning Concurrent Programming in Scala }
\dates{Packt Publishing 2014}
\newline
\noindent Aleksandar Prokopec
\bigskip


%% Publications
\marginhead{Publications}
% \medskip

%% Use revnumerate environment if numbered publications are needed.
%% (Include it above in the preamble).
%% \renewcommand{\labelenumi}{\textsc{a}\theenumi.}
%% \begin{revnumerate}

\noindent
Authored 12 international research publications
and several technical reports.
\newline


\noindent\href{http://axel22.github.io/resources/docs/reactive-isolates.pdf}
{\bf Isolates, Channels and Event Streams for Composable }
\dates{Onward! 2015}
\newline
\noindent\href{http://axel22.github.io/resources/docs/reactive-isolates.pdf}
{\bf Distributed Programming}
\dates{}
\newline\noindent Aleksandar Prokopec, Martin Odersky
\newline
\dates{Onward! 2015}
% \bigskip
\medskip

\noindent\href{http://axel22.github.io/resources/docs/lcpc-conc-trees.pdf}
{\bf Conc-Trees for Functional and Parallel Programming}
\dates{LCPC 2015}
\newline\noindent Aleksandar Prokopec, Martin Odersky
% \bigskip
\newline
\dates{Languages and Compilers for Parallel Computing 2015}
\medskip

\noindent\href{http://axel22.github.io/resources/docs/snapqueue.pdf}
{\bf SnapQueue: Lock-Free Queue with Constant Time Snapshots}
\dates{SCALA 2015}
\newline
\noindent Aleksandar Prokopec
\newline
\dates{Scala Symposium 2015, co-located with PLDI}
\bigskip

\noindent\href{http://axel22.github.io/resources/docs/pdp.pdf}
{\bf Efficient Lock-Free Work-stealing Iterators for Data-Parallel Collections}
\dates{PDP 2015}
\newline
\noindent Aleksandar Prokopec, Dmitry Petrashko, Martin Odersky
\newline
\dates{Parallel, Distributed and Network-Based Processing 2015}
\bigskip

\noindent\href{http://axel22.github.io/resources/docs/reactives-and-isolates.pdf}
{\bf Containers and Aggregates, Mutators and Isolates}
\dates{SCALA 2014}
\newline
\noindent\href{http://axel22.github.io/resources/docs/reactives-and-isolates.pdf}
{\bf for Reactive Programming}
\newline
\noindent Aleksandar Prokopec, Philipp Haller, Martin Odersky
\newline
\dates{Annual Scala Workshop 2014, co-located with ECOOP}
\bigskip

\noindent\href{http://axel22.github.io/resources/docs/lcpc2013_submission_6.pdf}
{\bf Near Optimal Work-Stealing Tree Scheduler }
\dates{LCPC 2013}
\newline
\noindent\href{http://axel22.github.io/resources/docs/lcpc2013_submission_6.pdf}
{\bf for Highly Irregular Data-Parallel Workloads}
\newline
\noindent Aleksandar Prokopec, Martin Odersky
\newline
\dates{Languages and Compilers for Parallel Computing 2013}
\bigskip

\noindent\href{http://axel22.github.io/resources/docs/ecoop13_sujeeth.pdf}
{\bf Composition and Reuse with Compiled Domain-Specific Languages}
\dates{ECOOP 2013}
\newline
\noindent Arvind K. Sujeeth, Tiark Rompf, Kevin J. Brown,
\newline
\noindent HyoukJoong Lee,
          Hassan Chafi, Victoria Popic, Michael Wu,
\newline
\noindent Aleksandar Prokopec,
          Vojin Jovanovic, Martin Odersky, Kunle Olukotun
\newline
\dates{European Conference on Object-Oriented Programming 2013}
\bigskip

\noindent\href{http://axel22.github.io/resources/docs/lcpc2012.pdf}
{\bf FlowPools: A Lock-Free Deterministic Concurrent }
\dates{LCPC 2012}
\newline
\noindent\href{http://axel22.github.io/resources/docs/lcpc2012.pdf}
{\bf Dataflow Abstraction}
\newline
\noindent Aleksandar Prokopec, Heather Miller, Tobias Schlatter
\newline
\noindent Philipp Haller, Martin Odersky
\newline
\dates{Languages and Compilers for Parallel Computing 2012}
\bigskip

\noindent\href{http://axel22.github.io/resources/docs/ctries-snapshot.pdf}
{\bf Concurrent Tries with Efficient Non-blocking Snapshots}
\dates{PPOPP 2012}
\newline
\noindent Aleksandar Prokopec, Nathan Bronson, Phil Bagwell, Martin Odersky
\newline
\dates{Symposium on Principles and Practice of Parallel Programming 2012}
\bigskip

\noindent\href{http://lampwww.epfl.ch/~prokopec/lcpc_ctries.pdf}
{\bf Lock-Free Resizeable Concurrent Tries}
\dates{LCPC 2011}
\newline
\noindent Aleksandar Prokopec, Phil Bagwell, Martin Odersky
\newline
\dates{Languages and Compilers for Parallel Computing 2011}
\bigskip

\noindent\href{http://infoscience.epfl.ch/record/165523/files/techrep.pdf}
{\bf A Generic Parallel Collection Framework}
\dates{Euro-Par 2011}
\newline
\noindent Aleksandar Prokopec, Phil Bagwell, Tiark Rompf, Martin Odersky
\newline
\dates{Euro-Par 2011}
\bigskip

\noindent\href{http://axel22.github.io/resources/docs/icadiwt_atga.pdf}
{\bf Adaptive Mutation Operator Cycling}
\dates{ICADIWT 2009}
\newline
\noindent Aleksandar Prokopec, Marin Golub
\newline
\dates{International Conference on the Applications }
\newline
\dates{of Digital Information and Web Technologies 2009}
\bigskip
\bigskip
\bigskip

%% In Progress
%\marginhead{{\vskip 0.4em}Under \newline Submission}
%\medskip

%% Use revnumerate environment if numbered publications are needed.
%% (Include it above in the preamble).
%% \renewcommand{\labelenumi}{\textsc{a}\theenumi.}
%% \begin{revnumerate}

%\noindent{\bf Cache Tries -- High-Performance Lock-Free Hashtables}\dates{}
%\newline\noindent Aleksandar Prokopec, Nikos Papakonstantinou, Polyvios Pratikakis
% \bigskip
%\medskip

%% Selected Tech Reports
\marginhead{{\vskip 0.3em}Selected \newline Tech Reports}
% \medskip

%% Use revnumerate environment if numbered publications are needed.
%% (Include it above in the preamble).
%% \renewcommand{\labelenumi}{\textsc{a}\theenumi.}
%% \begin{revnumerate}

\noindent\href{http://infoscience.epfl.ch/record/186071}
{\bf  Achieving Efficient Work-Stealing for Data-Parallel Collections}
\dates{April 2013}
% \smallskip
\newline\noindent Aleksandar Prokopec, Martin Odersky
% \bigskip
\medskip

\noindent\href{http://infoscience.epfl.ch/record/198208}
{\bf  Multi-Lane FlowPools: A Detailed Look}
\dates{September 2012}
\newline\noindent Tobias Schlatter, Aleksandar Prokopec, Heather Miller,
\newline\noindent Philipp Haller, Martin Odersky
\medskip

\noindent\href{http://infoscience.epfl.ch/record/181098}
{\bf  FlowPools: A Lock-Free Deterministic Concurrent Dataflow }
\dates{June 2012}
\newline
\noindent\href{http://infoscience.epfl.ch/record/181098}
{\bf  Abstraction -- Proofs}
\dates{}
% \smallskip
\newline\noindent Aleksandar Prokopec, Heather Miller, Philipp Haller
% \bigskip
\medskip

\noindent\href{http://infoscience.epfl.ch/record/166908/files/ctries-techreport.pdf}
{\bf Cache-Aware Lock-Free Concurrent Hash Tries}
\dates{June 2011}
% \smallskip
\newline\noindent Aleksandar Prokopec, Phil Bagwell, Martin Odersky
% \bigskip
\medskip

%% Open Source
\medskip
\marginhead{{\vskip 0.1em}Open Source}

\vspace{0.01in}
\noindent {\bf Scala Programming Language}, {\em Scala team member} \dates{2009 -- now}

\vspace{0.05in}
\begin{easylist}[itemize]
& \href{http://storm-enroute.com/coroutines}
{{\bf Scala Coroutines}},
{\bf \em lead}
\newline Scala language extension for first-class coroutines,
\newline used for easier asynchronous programming.

& \href{http://reactors.io/}
{{\bf Reactors Framework for Distributed Programming}},
{\bf \em lead}
\newline Programming framework aimed at building asynchronous,
\newline composable, distributed systems.

& \href{http://scala-blitz.github.io/}
{{\bf Scala-Blitz -- High-Performance Data-Parallelism Framework}},
{\bf \em lead}
\newline Scala module for highly efficient data-parallel programming.

& \href{http://scalameter.github.io/}
{{\bf ScalaMeter Benchmarking Framework}},
{\bf \em lead}
\newline Microbenchmarking and performance regression testing framework
\newline for Scala and JVM, adopted in various open source projects.

& \href{http://docs.scala-lang.org/sips/completed/futures-promises.html}
{{\bf Scala Futures \& Promises} (Scala Improvement Proposal 14)},
{\bf \em team member}
\newline Asynchronous programming framework for Scala, used as a
\newline basic building block for other concurrency frameworks within
\newline the Scala ecosystem (part of standard Scala distribution since 2012).

& \href{http://docs.scala-lang.org/overviews/parallel-collections/overview.html}
{{\bf Parallel Collections Framework}},
{\bf \em lead}
\newline Scala standard library module for data-parallel programming
\newline (part of the standard Scala distribution in 2011).
\end{easylist}

\vspace{0.10in}
\noindent {\bf Java Group at Faculty of Electrical Engineering and Computing, Zagreb},
\dates{2006 -- 2008}

\begin{easylist}
& \href{https://github.com/mbezjak/vhdllab}
{{\bf VHDLLab}},
{\bf \em team member}
\newline Award-winning online educational VHDL editor for modeling and
\newline simulation of digital circuits (used as part of the computer
\newline science curriculum at the Faculty of Electrical Engineering
\newline and Computing in Zagreb since 2007).
\end{easylist}

\bigskip

%% Honors
\medskip
\medskip
\medskip

\marginhead{Honors and \newline Awards}

\noindent Google Spot Bonus for the Distributed Real-Time Spam Detection Project
\dates{2015}
\newline\noindent Google Peer Bonus for the Team Hackathon Initiative \dates{2015}
\newline\noindent Nominated for the Patrick Denantes Doctoral Thesis Award \dates{2014}
\newline\noindent EPFL Outstanding Teaching Assistant Award \dates{2013}
\newline\noindent LCPC Best Paper Presentation Award \dates{2011}
\newline\noindent EPFL Computer Science Fellowship \dates{2009 -- 2010}
\newline\noindent University of Zagreb Rector Award for Best Project (VHDLLab)
                  \dates{2008}
\newline\noindent Faculty of Electrical Engineering and Computing Josip Loncar Award
                  \dates{2007}
\newline\noindent Faculty of Electrical Engineering and Computing Josip Loncar Award
                  \dates{2006}
\newline\noindent Participation in the International Physics Olympiad (IPhO)
                  \dates{2004}
\newline\noindent 1st Place in the Croatian National Physics Competition (Finals)
\dates{2004}
\newline\noindent Participation in the Croatian National Physics Competition (Finals)
\dates{2002}
\newline\noindent Participation in the Croatian National Physics Competition (Finals)
\dates{2001}

\bigskip




%% Talks
\medskip
\marginhead{Selected Talks}

\vspace{-0.02in}

\noindent
Gave over 20 academic and industrial talks, 6 as an invited speaker.
\newline

\noindent\href{https://cfp-vdz.exteso.com/program/speaker/aleksandar_prokopec.html}
{\bf Reactor Model for Composable Distributed Computing }
\dates{Voxxed Days Zurich}\vspace{-0.03in}
\newline\noindent Developer Conference Talk
\dates{}
\linebreak\noindent Zurich, Switzerland, March 3, 2016
\bigskip

\noindent\href{http://axel22.github.io/resources/docs/onward15.pdf}
{\bf Isolates, Channels and Event Streams for Composable }
\dates{Onward! 2015}\vspace{-0.03in}
\newline
\noindent\href{http://axel22.github.io/resources/docs/onward15.pdf}
{\bf Distributed Programming }
\newline\noindent Academic Conference Talk
\dates{}
\linebreak\noindent Pittsburgh, Pennsylvania, USA, October 29, 2015
\bigskip

\noindent\href{http://axel22.github.io/resources/docs/jazoon15.pdf}
{\bf Scala -- The Learning Curve}
\dates{Jazoon 2015}\vspace{-0.03in}
\newline\noindent Developer Conference Talk
\dates{}
\linebreak\noindent Zurich, Switzerland, October 23, 2015
\bigskip

\noindent\href{http://axel22.github.io/slides/conc.html#/}
{\bf Conc-Tree Data Structure for Functional and Parallel Programming}
\dates{LCPC 2015}\vspace{-0.03in}
\newline\noindent Academic Conference Talk
\dates{}
\linebreak\noindent Raleigh, North Carolina, USA, September 10, 2015
\bigskip

\noindent\href{http://axel22.github.io/slides/snapq.html#/}
{\bf SnapQueue: Lock-Free Queue with Constant Time Snapshots}
\dates{SCALA 2015}\vspace{-0.03in}
\newline\noindent Academic Conference Talk
\dates{}
\linebreak\noindent Portland, Oregon, USA, June 13, 2015
\bigskip

\noindent\href{https://speakerdeck.com/axel22/scalameter-in-2014}
{\bf ScalaMeter -- Performance Regression Testing Framework}
\dates{Oracle VM Meetup 2014}\vspace{-0.03in}
\newline\noindent Academic Meetup Talk
\dates{}
\linebreak\noindent ETH, Zürich, Switzerland, September 11, 2014
\bigskip

\noindent
{\bf Containers and Aggregates, Mutators and Isolates }
\dates{SCALA 2014}\vspace{-0.03in}
\newline
{\bf for Reactive Programming}
\dates{}
\newline\noindent Academic Conference Talk
\dates{}
\linebreak\noindent Uppsala, Sweden, July 28, 2014
\bigskip

\noindent\href{https://www.parleys.com/tutorial/53a7d2cde4b0543940d9e561/chapter1/about}
{\bf Reactive Collections and 3D Engine Design}
\dates{ScalaDays 2014}\vspace{-0.03in}
\newline\noindent Industrial Conference Talk (800 attendees)
\dates{}
\linebreak\noindent Berlin, Germany, June 24, 2014
\bigskip

\noindent
\href{http://skillsmatter.com/podcast/scala/macro-based-scala-parallel-collections}
{\bf Macro-based Scala Parallel Collections}
\dates{Scala eXchange 2013}\vspace{-0.03in}
\newline\noindent Industrial Conference Talk \textbf{(350 attendees, invited talk)}
\dates{}
\linebreak\noindent London, UK, December 2, 2013
\bigskip

\noindent\href{http://axel22.github.io/resources/docs/lcpc13.pptx}
{\bf Work-Stealing Tree Scheduling}
\dates{LCPC 2013}\vspace{-0.03in}
\newline\noindent Academic Conference Talk
\dates{}
\linebreak\noindent San Jose, CA, USA, September 26, 2013
\bigskip

\noindent
{\bf Scala as a Research Tool}
\dates{ECOOP 2013}\vspace{-0.03in}
\newline\noindent ECOOP Tutorial
                  (w/ Heather Miller and Philipp Haller)
\dates{}
\linebreak\noindent Montpellier, France, July 1, 2013
\bigskip

\noindent
\href{https://skillsmatter.com/skillscasts/3701-scala-performance-regression-testing}
{\bf Scala Performance Regression Testing}
\dates{Scala eXchange 2012}\vspace{-0.03in}
\newline\noindent Industrial Conference Talk \textbf{(invited talk)}
\dates{}
\linebreak\noindent London, UK, November 19, 2012
\bigskip

\noindent\href{https://skillsmatter.com/skillscasts/3234-parallel-concurrent-hash-tries}
{\bf Parallel Concurrent Hash Tries}
\dates{ScalaDays 2012}\vspace{-0.03in}
\newline\noindent Industrial Conference Talk (400 attendees)
\dates{}
\linebreak\noindent London, UK, April 18, 2012
\bigskip

\noindent
{\bf Concurrent Hash Tries in Scala}
\dates{Croatian IEEE Chapter Meetup}\vspace{-0.03in}
\newline\noindent Academic Meetup Talk \textbf{(invited talk)}
\dates{}
\linebreak\noindent Zagreb, Croatia, April 11, 2012
\bigskip

\noindent\href{http://axel22.github.io/resources/docs/ctries-snapshots.pptx}
{\bf Concurrent Tries with Efficient Non-Blocking Snapshots}
\dates{PPOPP 2012}\vspace{-0.03in}
\newline\noindent Academic Conference Talk
\dates{}
\linebreak\noindent New Orleans, Louisiana, USA, February 29, 2012
\bigskip

\noindent
\href{http://lcpc11.cs.colostate.edu/}
{\bf Lock-Free Resizeable Concurrent Tries}
\dates{LCPC 2011}\vspace{-0.03in}
\newline\noindent Academic Conference Talk
\dates{}
\linebreak\noindent Fort Collins, CO, USA, September 9, 2011
\bigskip

\noindent
\href{http://europar2011.bordeaux.inria.fr/conference.php}
{\bf Generic Parallel Collection Framework}
\dates{Euro-Par 2011}\vspace{-0.03in}
\newline\noindent Academic Conference Talk
\dates{}
\linebreak\noindent Bordeaux, France, September 2, 2011
\bigskip

\noindent\href{https://skillsmatter.com/skillscasts/2236-parallel-collections}
{\bf Parallel Collections}
\dates{Scala eXchange 2011}\vspace{-0.03in}
\newline\noindent Industrial Conference Talk \textbf{(280 attendees, invited talk)}
\dates{}
\linebreak\noindent London, UK, June 15, 2011
\bigskip

\noindent
\href{http://days2011.scala-lang.org/node/138/272/29.%20Parallel%20Collections.mp4}
{\bf Scala Parallel Collections}
\dates{ScalaDays 2011}\vspace{-0.03in}
\newline\noindent Industrial Conference Talk
\dates{}
\linebreak\noindent Palo Alto, CA, USA, June 3, 2011
\bigskip

\noindent
\href{http://www.slideshare.net/AleksandarProkopec/introduction-to-scala-39540464}
{\bf Introduction to Scala}
\dates{JUG Meetup 2011}\vspace{-0.03in}
\newline\noindent Industrial Talk, Java User Group Grenoble \textbf{(invited talk)}
\dates{}
\linebreak\noindent Grenoble, France, March, 2011
\bigskip

\noindent\href{http://days2010.scala-lang.org/node/138/140/}
{\bf Parallel Collections}
\dates{ScalaDays 2010}\vspace{-0.03in}
\newline\noindent Industrial Conference Talk \textbf{(150 attendees, invited talk)}
\dates{}
\linebreak\noindent Lausanne, Switzerland, April 15, 2010
\bigskip



\bigskip


% %% Selected Broader Service
% \medskip
% \marginhead{Selected \newline Broader \newline Service}

% \noindent \href{http://ic.epfl.ch/conseil-de-faculte}{\bf EPFL Computer Science Faculty Council}, {\bf \em PhD Student Representative} \dates{2012 --}
% \newline\noindent Members include the dean of the faculty as well as representatives
% \newline\noindent from every branch of the faculty, administrative, PhD, faculty, etc.
% \newline\noindent Quarterly meetings to steer the faculty and introduce new initiatives.
% \bigskip

% \noindent \href{http://ic-gsa.epfl.ch/}{\bf EPFL CS Graduate Student Association}, {\bf \em President} \dates{2009 -- 2011}
% \newline\noindent Volunteer student organization with a mission to foster a sense of
% \newline\noindent community and collaboration between different research groups in
% \newline\noindent the faculty. Initiatives led/introduced:
% \vspace{0.05in}
% \begin{easylist}[itemize]
% & {\bf Research Day}: college-wide showcase of labs' research activities
% & {\bf PhD Student Open House}: main recruiting event for CS doctoral program
% & {\bf Social Events}: aper\'{o}s, ski trips, outings
% \end{easylist}
% \bigskip

% \noindent {\bf EPFL CS Graduate Student Mentor} \dates{2010 -- 2012}
% \newline\noindent One-on-one mentoring of incoming doctoral students, aided students in
% \newline\noindent integrating into EPFL's research environment and Switzerland as a whole.
% \vspace{0.05in}
% \bigskip


%% External Service
\medskip
\marginhead{External \newline Service}

\noindent
Reviewer or program committee member of 8 scientific conferences and journals.

% \noindent {\bf Committees:}
\noindent\newline\noindent {\bf Programming based on Actors, Agents, and Decentralized Control (AGERE! 2016)},
program committee \dates{2016}
\noindent\newline\noindent {\bf International Conference on Parallel and Distributed Systems (ICPADS 2016)},
program committee \dates{2016}
\noindent\newline\noindent {\bf High-Level Parallel Programming and Applications (HLPP 2016)},
program committee \dates{2016}
\noindent\newline\noindent {\bf On Principles of Distributed Systems (OPODIS 2015)},
external reviewer \dates{2015}
\noindent\newline\noindent {\bf Transactions on Computers 2015},
journal submission reviewer \dates{2015}
\newline\noindent {\bf Scala Workshop 2014},
program committee \dates{7/2014}
\newline\noindent {\bf High-Level Parallel Programming and Applications 2014},
external reviewer \dates{4/2014}
\newline\noindent {\bf ECOOP 2013},
external reviewer \dates{7/2013}
\newline\noindent {\bf Scala Workshop 2013},
program committee (co-chair)\dates{7/2013}
\newline\noindent {\bf ScalaDays 2010},
external reviewer \dates{6/2010}
\newline\noindent {\bf ICADIWT 2009},
external reviewer \dates{8/2009}

\bigskip

% %% Service
% \marginhead{Academic \newline Service}

% \vspace{-0.02in}
% \noindent{\bf Committees}: Curry On Prague (co-chair), Scala 2015 (co-chair), ECOOP 2015 organizing committee (sponsorship chair), POPL 2015 AEC, Scala 2014 (co-chair), Scala 2013 (co-chair)
% \newline\noindent{\bf Reviewer} for: ECOOP 2013, Scala 2013

% \bigskip

%% Students Supervised
\medskip
\marginhead{Supervised \newline Projects\footnotemark[1]}
\footnotetext[1]{EPFL research labs prepare projects for B.Sc./M.Sc.
students to complete for credits.
These projects are designed and supervised by EPFL doctoral assistants.}

\noindent
Supervised 14 bachelor and master student projects.
\newline

\noindent
{\bf Joël Rossier},
{\em MacroGL Scala.JS Backend}
\dates{2/2014 -- 6/2014}
\newline\noindent B.Sc. level
\medskip

\noindent
{\bf Sven Reber},
{\em MacroGL API Extensions}
\dates{2/2014 -- 6/2014}
\newline\noindent B.Sc. level
\medskip

\noindent
{\bf Gwangbae Choi},
{\em ScalaMeter Inline Benchmarking}
\dates{2/2014 -- 6/2014}
\newline\noindent B.Sc. level
\medskip

\noindent
{\bf Kristof Szabo},
{\em ScalaMeter Java API}
\dates{2/2014 -- 6/2014}
\newline\noindent B.Sc. level
\medskip

\noindent
{\bf Nicolas Stucki},
{\em Scala Multiset Collection}
\dates{9/2013 -- 1/2014}
\newline\noindent M.Sc. level
\medskip

\noindent
{\bf Timo Babst},
{\em Data-Parallel Raytracer}
\dates{9/2013 -- 1/2014}
\newline\noindent B.Sc. level
\medskip

\noindent
{\bf Clément Moutet},
{\em Data-Parallel Flocking Algorithm}
\dates{9/2013 -- 1/2014}
\newline\noindent B.Sc. level
\medskip

\noindent
{\bf Roman Zoller},
{\em ScalaMeter D3js Frontend}
\dates{2/2013 -- 6/2013}
\newline\noindent M.Sc. level
\medskip

\noindent
{\bf Tobias Schlatter},
{\em FlowSeqs: Barrier-Free ParSeqs}
\dates{9/2012 -- 1/2013}
\newline\noindent M.Sc. level, co-supervision w/ Philipp Haller \& Heather Miller
\medskip

\noindent
{\bf Roger Vion},
{\em Improvements to ScalaMeter}
\dates{9/2012 -- 1/2013}
\newline\noindent B.Sc. level
\medskip

\noindent
{\bf Tobias Schlatter},
{\em Multi-Lane FlowPools}
\dates{2/2012 -- 6/2012}
\newline\noindent M.Sc. level, co-supervision w/ Philipp Haller \& Heather Miller
\medskip

\noindent
{\bf Bruno Studer},
{\em A Non-Blocking Concurrent Queue Algorithm}
\dates{2/2012 -- 6/2012}
\newline\noindent B.Sc. level
\medskip

\noindent
{\bf Ngoc Duy Pham},
{\em Scala Benchmarking Suite -- Performance}
\dates{8/2011 -- 1/2012}
\newline\noindent {\em Regression Pinpointing}
\newline\noindent M.Sc. level
\medskip

\noindent
{\bf Pamela Delgado},
{\em Scala Invariant Verifier}
\dates{8/2011 -- 1/2012}
\newline\noindent M.Sc. level
\medskip

\bigskip

\medskip
\marginhead{GSoC \newline Projects\footnotemark[2]}
\footnotetext[2]{The Scala Team is a regular host of Google Summer of Code Projects.
These are 3-month paid projects offered to students all over the world,
sponsored by Google and supervised by members of various open-source organizations.}


\noindent
Supervised 4 Google Summer of Code projects.
\newline

\noindent
{\bf Krzysztof Janosz},
{\em ScalaMeter –- Binary Compatible Serialization }
\dates{5/2015 -- 9/2015}
\newline\noindent
{\em Format and Invocation Measurers}
\dates{}
\medskip

\noindent
{\bf Dmitry Petrashko},
{\em Specializing Parallel Collections with Scala Macros }
\dates{5/2013 -- 9/2013}
\medskip

\noindent
{\bf Ivan Oreskovic},
{\em Porting Scala Parallel Collections to the }
\dates{5/2012 -- 9/2012}
\newline\noindent
{\em Android Platform}
\dates{}
\medskip

\noindent
{\bf Heather Miller},
{\em Parallel Collections Extensions}
\dates{5/2011 -- 9/2011}
\medskip


\pagebreak
%% References
\medskip
\marginhead{References}

\noindent {\bf Martin Odersky}
\newline\noindent {Faculty of Computer, Communication, and Information Science}
\newline\noindent {\em \'{E}cole Polytechnique F\'{e}d\'{e}rale de Lausanne}
\newline\noindent \faPhone~+41 21 693 68 63
\newline\noindent \Letter~\href{mailto:martin.odersky@epfl.ch}{martin.odersky@epfl.ch}
\medskip

\noindent {\bf Philipp Haller}
\newline\noindent {School of Computer Science and Communication}
\newline\noindent {\em KTH Royal Institute of Technology}
\newline\noindent \faPhone~+41 76 205 39 32
\newline\noindent \Letter~\href{mailto:phaller@kth.se}{phaller@kth.se}
\medskip



\end{document}
